\section{給付}

%4節 DBの給付設計
\subsubsection{リスク分担型企業年金の調整率}
\barquo{
  リスク分担型企業年金を実施する事業主等がその実施事業所を減少させるときに、当該実施
  事業所の減少に伴い資格を喪失する加入者に係る調整率を別に定めることができる場合とし
  て、確定給付企業年金法施行規則第25条の2第2項に定める場合を簡潔に入力しなさい。
  (250 字以内)

  \rightline{引用元:年金1 2023 問2(1)(イ)}
}

\begin{itembox}[l]{\textgt{ポイント}}
  リスク分担型企業年金では、実施事業所が減少する際に
  掛金収入現価、通常予測給付現価、財政悪化リスク相当額が減少することで
  財政バランスが崩れ、現在または将来の調整率が変化する可能性がある。
  このようなケースで、\textcolor{red}{積立割合、調整率または超過比率}が減少すると見込まれる場合、
  積立割合、調整率または超過比率が減少しないように、
  実施事業所の減少に伴い資格を喪失する加入者にかかる調整率を別に定めることが可能とされている。
  (則25条の2第2項)
\end{itembox}

\begin{sol}

  \;

  当該実施事業所の減少に伴い、積立比率、調整率または超過比率が減少すると見込まれる場合。
\end{sol}

\newpage

\subsubsection{ポイントの満たすべき要件}
\barquo{
  確定給付企業年金法第32条第2項において、給付の額は、加入者期間又は当該加入者期間
  における給与の額その他これに類するものに照らし、適正かつ合理的な方法により算定され
  たものでなければならないと定められている。

  「その他これに類するもの」は、ポイント制を採用している場合における労働協約等に定め
  られたポイントをいうが、当該ポイントが満たすべき要件として、 『確定給付企業年金法並び
  にこれに基づく政令及び省令について(法令解釈)』に掲げる要件を3つ簡潔に入力しなさ
  い。 (250 字以内)

  \rightline{引用元:年金1 2023 問2(1)(エ)}
}

\begin{itembox}[l]{\textgt{ポイント}}
  解釈3-1\textcircled{4}参照。
\end{itembox}

\begin{sol}

  \;

  \begin{itemize}
    \item 昇格の規定が労働協約等において明確に定められていること。
    \item 同一の加入者期間を有する加入者について、最大ポイントの最小ポイントに対する割合に過大な格差がないこと。
    \item ポイントは恣意的に決められるものではなく、数理計算が可能であること
  \end{itemize}
  
\end{sol}

\begin{shadebox}
  「最大ポイントの最小ポイントに対する割合に過大な格差がないこと」
  という規定内の「過大な格差」の数値的基準として、昔は「原則として15倍以内」というルールが定められいた。
  (平成21年3月3日改正以前)。

  今ではこの「15倍ルール」は撤廃されている。
\end{shadebox}

\newpage