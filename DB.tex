\documentclass[10pt]{jsarticle}%文字サイズが10ptのjsarticle

\input{preamble.tex}


\setcounter{tocdepth}{3}%目次に含めるレベル。1ならsectionまで。2ならsubsectionまで。3ならsubsubsectionまで。
\begin{document}



\title{年金2次 \; 簡記問題集 \; DB編}
\author{hyaraguchi}
\date{\today}
\maketitle


\tableofcontents%目次

\newpage

%2章 DBの開始
\section{DBの開始}

%2節 1事業所1DBの原則
\subsubsection{1事業所1DBの原則}
\barquo{
  1つの厚生年金適用事業所が複数の確定給付企業年金を実施することができる場合として、
  確定給付企業年金法施行令第1条に定める場合(注)以外の、確定給付企業年金法施行規則
  第1条に定める場合を3つ簡潔に入力しなさい。 (250 字以内)

  (注)複数の厚生年金適用事業所が共同で実施する確定給付企業年金に加入する一方で、企業
  独自に実施する確定給付企業年金に加入する等、企業年金を実施する事業主の範囲が異
  なる場合。
  
  \rightline{引用元:年金1 2023 問2(1)(ア)}
}

\begin{itembox}[l]{\textgt{ポイント}}
  DBは、基本的に1つの適用事業所につき1つしか実施することができない。
  しかし、法令に定める限りにおいて、2つ以上DBを実施できる場合がある。
  (法第3条第2項)

  令第1条、則第1条に「1事業所1DBの原則」の例外が限定列挙されている。
  \begin{enumerate}
    \item 2つのDBのうちいずれか一方の実施事業所の事業主の全部が、他方のDBの事業主の全部とならないとき($=$\textcolor{red}{総合型DB})
    \item それぞれの加入者に適用される\textcolor{red}{労働協約、就業規則その他これらに類するもの}が異なる場合
    \item 法人である事業主が他の法人である事業主と合併した場合であって、当該合併の日から起算して原則として\textcolor{red}{1年}を超過しない場合
    \item \textcolor{red}{リスク分担型企業年金}とリスク分担型企業年金でないDBとをそれぞれ実施する場合
  \end{enumerate}

  1が令1条に、2〜4が則1条に記載されている。
  令1条のケースは問題文に記載されているので、この問題では、2〜4について簡記すればよい。

\end{itembox}

\begin{sol}

  \;

  \begin{itemize}
    \item それぞれの確定給付企業年金の加入者に適用される労働協約等が異なる場合
    \item 法人の実施主同士が合併した日から起算して原則1年を超過しない場合
    \item リスク分担型企業年金とリスク分担型企業年金ではない確定給付企業年金をそれぞれ実施する場合
  \end{itemize}

\end{sol}



\newpage


%3章 加入者
\section{加入者}

%1節 加入者範囲と一定の資格
\subsubsection{加入者資格のグループ区分}
\barquo{
  加入者の資格を区分(グループ区分)することができる場合として、『確定給付企業年金法並び
  にこれに基づく政令及び省令について(法令解釈)』に定める場合を簡潔に入力しなさい。
  (250 字以内)

  \rightline{引用元:年金1 2023 問2(1)(ウ)}
}

\begin{itembox}[l]{\textgt{ポイント}}
  解釈第1-1(3)の内容をそのまま書けばよい。

  
\end{itembox}

\begin{sol}

  \;

  労働協約等における給与および退職金等の労働条件が異なるなど
  合理的な理由がある場合
\end{sol}

\begin{shadebox}

  則第1条の記載の通り、労働条件が異なる場合は別のDBを実施することもできる。

\end{shadebox}

\newpage

%4章 給付
\section{給付}

%4節 DBの給付設計
\subsubsection{リスク分担型企業年金の調整率}
\barquo{
  リスク分担型企業年金を実施する事業主等がその実施事業所を減少させるときに、当該実施
  事業所の減少に伴い資格を喪失する加入者に係る調整率を別に定めることができる場合とし
  て、確定給付企業年金法施行規則第25条の2第2項に定める場合を簡潔に入力しなさい。
  (250 字以内)

  \rightline{引用元:年金1 2023 問2(1)(イ)}
}

\begin{itembox}[l]{\textgt{ポイント}}
  リスク分担型企業年金では、実施事業所が減少する際に
  掛金収入現価、通常予測給付現価、財政悪化リスク相当額が減少することで
  財政バランスが崩れ、現在または将来の調整率が変化する可能性がある。
  このようなケースで、\textcolor{red}{積立割合、調整率または超過比率}が減少すると見込まれる場合、
  積立割合、調整率または超過比率が減少しないように、
  実施事業所の減少に伴い資格を喪失する加入者にかかる調整率を別に定めることが可能とされている。
  (則25条の2第2項)
\end{itembox}

\begin{sol}

  \;

  当該実施事業所の減少に伴い、積立比率、調整率または超過比率が減少すると見込まれる場合。
\end{sol}

\newpage

\subsubsection{ポイントの満たすべき要件}
\barquo{
  確定給付企業年金法第32条第2項において、給付の額は、加入者期間又は当該加入者期間
  における給与の額その他これに類するものに照らし、適正かつ合理的な方法により算定され
  たものでなければならないと定められている。

  「その他これに類するもの」は、ポイント制を採用している場合における労働協約等に定め
  られたポイントをいうが、当該ポイントが満たすべき要件として、 『確定給付企業年金法並び
  にこれに基づく政令及び省令について(法令解釈)』に掲げる要件を3つ簡潔に入力しなさ
  い。 (250 字以内)

  \rightline{引用元:年金1 2023 問2(1)(エ)}
}

\begin{itembox}[l]{\textgt{ポイント}}
  解釈3-1\textcircled{4}参照。
\end{itembox}

\begin{sol}

  \;

  \begin{itemize}
    \item 昇格の規定が労働協約等において明確に定められていること。
    \item 同一の加入者期間を有する加入者について、最大ポイントの最小ポイントに対する割合に過大な格差がないこと。
    \item ポイントは恣意的に決められるものではなく、数理計算が可能であること
  \end{itemize}
  
\end{sol}

\begin{shadebox}
  「最大ポイントの最小ポイントに対する割合に過大な格差がないこと」
  という規定内の「過大な格差」の数値的基準として、昔は「原則として15倍以内」というルールが定められいた。
  (平成21年3月3日改正以前)。

  今ではこの「15倍ルール」は撤廃されている。
\end{shadebox}

\newpage




\begin{comment}
\newpage

\begin{thebibliography}{1}%参考文献の リスト
  \bibitem[過去問]{過去問} 公益社団法人 日本アクチュアリー会 資格試験過去問題集 \url{https://www.actuaries.jp/lib/collection/} (最終閲覧日:2023/12/10)
  \bibitem[教科書]{教科書} 日本アクチュアリー会『損保数理』(日本アクチュアリー会, 2011)
  \bibitem[モデリング]{モデリング} 日本アクチュアリー会『モデリング』(日本アクチュアリー会, 2005)
  \bibitem[リスク・セオリー]{リスク・セオリー} 岩沢宏和『リスク・セオリーの基礎』(培風館, 2010)
  \bibitem[アク数学シリーズ]{アク数学シリーズ} 岩沢宏和, 黒田耕嗣『アクチュアリー数学シリーズ4 損害保険数理』(日本評論社, 2015)
  \bibitem[ストラテジー]{ストラテジー} MAH, 平井卓也, 玉岡一史『アクチュアリー試験 合格へのストラテジー 損保数理』(東京図書, 2019)
  \bibitem[例題で学ぶ]{例題で学ぶ} 小暮雅一, 東出純『例題で学ぶ損害保険数理 第2版』(共立出版, 2016)
  \bibitem[難問題の系統]{難問題の系統}CAR他「難問題の系統とその解き方 損保数理」\url{}(最終閲覧日:2023/12/10)
  \bibitem[弱点克服]{弱点克服} 藤田岳彦『弱点克服 大学生の確率・統計』(東京図書, 2010)
  \bibitem[数研微積分]{数研微積分} 加藤文元『大学教養 微分積分』(数研出版, 2019)
\end{thebibliography}
\end{comment}



\end{document}
