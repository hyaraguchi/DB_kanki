\section{DBの開始}

%2節 1事業所1DBの原則
\subsubsection{1事業所1DBの原則}
\barquo{
  1つの厚生年金適用事業所が複数の確定給付企業年金を実施することができる場合として、
  確定給付企業年金法施行令第1条に定める場合(注)以外の、確定給付企業年金法施行規則
  第1条に定める場合を3つ簡潔に入力しなさい。 (250 字以内)

  (注)複数の厚生年金適用事業所が共同で実施する確定給付企業年金に加入する一方で、企業
  独自に実施する確定給付企業年金に加入する等、企業年金を実施する事業主の範囲が異
  なる場合。
  
  \rightline{引用元:年金1 2023 問2(1)(ア)}
}

\begin{itembox}[l]{\textgt{ポイント}}
  DBは、基本的に1つの適用事業所につき1つしか実施することができない。
  しかし、法令に定める限りにおいて、2つ以上DBを実施できる場合がある。
  (法第3条第2項)

  令第1条、則第1条に「1事業所1DBの原則」の例外が限定列挙されている。
  \begin{enumerate}
    \item 2つのDBのうちいずれか一方の実施事業所の事業主の全部が、他方のDBの事業主の全部とならないとき($=$\textcolor{red}{総合型DB})
    \item それぞれの加入者に適用される\textcolor{red}{労働協約、就業規則その他これらに類するもの}が異なる場合
    \item 法人である事業主が他の法人である事業主と合併した場合であって、当該合併の日から起算して原則として\textcolor{red}{1年}を超過しない場合
    \item \textcolor{red}{リスク分担型企業年金}とリスク分担型企業年金でないDBとをそれぞれ実施する場合
  \end{enumerate}

  1が令1条に、2〜4が則1条に記載されている。
  令1条のケースは問題文に記載されているので、この問題では、2〜4について簡記すればよい。

\end{itembox}

\begin{sol}

  \;

  \begin{itemize}
    \item それぞれの確定給付企業年金の加入者に適用される労働協約等が異なる場合
    \item 法人の実施主同士が合併した日から起算して原則1年を超過しない場合
    \item リスク分担型企業年金とリスク分担型企業年金ではない確定給付企業年金をそれぞれ実施する場合
  \end{itemize}

\end{sol}



\newpage
