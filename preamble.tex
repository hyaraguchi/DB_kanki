%%%%%%%%%%%%%%%%%%%%%%%%%%%%%%%%%%%%%%%%%%%%%%%%%%%%%%%
%%  パッケージ                                        %%
%%%%%%%%%%%%%%%%%%%%%%%%%%%%%%%%%%%%%%%%%%%%%%%%%%%%%%%
%使用しないときはコメントアウトしてください
\usepackage{amsthm}%定理環境
\usepackage{framed}%文章を箱で囲う
\usepackage{amsmath,amssymb}%数式全般
\usepackage{ascmac} % itembox環境 
\usepackage[dvipdfmx]{graphicx}%図の挿入
%\usepackage{tikz}%描画
\usepackage{titlesec}%見出しの見た目を編集できる
\usepackage[dvipdfmx, usenames]{color}%色をつける
%\usepackage{tikz-cd}%可換図式
%\usepackage{mathtools}%数式関連
\usepackage{amsfonts}%数式のフォント
\usepackage[all]{xy}%可換図式
\usepackage{mathrsfs}%花文字
\usepackage{comment}%コメント環境
\usepackage{picture}%お絵かき
\usepackage{url}%URLを出力
\usepackage{ulem}% 波線、打ち消し線
\usepackage{fancybox}%文章の囲み


%%% ↓コメント外すとリンクに色がつく
%\usepackage[dvipdfmx]{hyperref}
%\usepackage{pxjahyper}%日本語しおりの文字化けを防ぐ

\allowdisplaybreaks[1]%数式がページをまたぐことを許す

%ページ番号を右上に表記
\makeatletter
\def\@evenhead{\hfil\thepage}%
\def\@oddhead{\hfil\thepage}%
\def\@evenfoot{\@empty}%
\def\@oddfoot{\@empty}%
\makeatother


%%%%%%%%%%%%%%%%%%%%%%%%%%%%%%%%%%%%%%%%%%%%%%%%%%%%%%%
%%  表紙                                             %%
%%%%%%%%%%%%%%%%%%%%%%%%%%%%%%%%%%%%%%%%%%%%%%%%%%%%%%%
\makeatletter
\def\thickhrulefill{\leavevmode \leaders \hrule height 1pt\hfill \kern \z@}
\renewcommand{\maketitle}{\begin{titlepage}%
    \let\footnotesize\small
    \let\footnoterule\relax
    \parindent \z@
    \reset@font
    \null\vfil
    \begin{flushleft}
      \huge \@title
    \end{flushleft}
    \par
    \hrule height 4pt
    \par
    \begin{flushright}
      \LARGE \@author \par
    \end{flushright}
    \vskip 60\p@
    \vfil\null
    \begin{flushright}
        {\small \@date}%
    \end{flushright}
  \end{titlepage}%
  \setcounter{footnote}{0}%
}
\makeatother




%%%%%%%%%%%%%%%%%%%%%%%%%%%%%%%%%%%%%%%%%%%%%%%%%%%%%%%
%%  sectionの修飾                                     %%
%%%%%%%%%%%%%%%%%%%%%%%%%%%%%%%%%%%%%%%%%%%%%%%%%%%%%%%
%四角ではじめて下線付き
\titleformat{\section}[block]
{}{}{0pt}
{
  \colorbox{black}{\begin{picture}(0,10)\end{picture}}
  \hspace{0pt}
  \normalfont \Large\bfseries
  \hspace{-4pt}
}
[
\begin{picture}(100,0)
  \put(3,18){\color{black}\line(1,0){300}}
\end{picture}
\\
\vspace{-30pt}
]

%subsubsectionの編集
\renewcommand{\thesubsubsection}{\textbf{問\arabic{subsubsection}}}%問(さぶさぶせくしょん番号)を出力






%%%%%%%%%%%%%%%%%%%%%%%%%%%%%%%%%%%%%%%%%%%%%%%%%%%%%%%
%%  番号付き定理環境                                  %%
%%%%%%%%%%%%%%%%%%%%%%%%%%%%%%%%%%%%%%%%%%%%%%%%%%%%%%%
%注:defというコマンドはもうある
\theoremstyle{definition}%定理環境のアルファベットを斜体にしない
\renewcommand{\proofname}{\textgt{証明}}%proof環境の修正

%%%%%%%%%%%%%%%%%%%%%%%%%%%%%%%%%%%%%%%%%%%%%%%%%%%%%%%
%%  番号なし定理環境                                  %%
%%%%%%%%%%%%%%%%%%%%%%%%%%%%%%%%%%%%%%%%%%%%%%%%%%%%%%%
%一部箱付き
\newtheorem*{lemma}{補題}
\newtheorem*{proposition}{命題}
\newtheorem*{definition}{定義}
\newcommand{\lem}[1]{\begin{oframed} \begin{lemma} #1 \end{lemma} \end{oframed}}%箱付きほだい
\newcommand{\prop}[1]{\begin{oframed} \begin{proposition} #1 \end{proposition} \end{oframed}}%箱付きめいだい

\newtheorem*{com}{コメント}
\newtheorem*{claim}{主張}
\newtheorem*{sol}{解答}
\newtheorem*{prob}{問題}
\newtheorem*{quo}{引用}
\newtheorem*{rem}{注意}



%%%%%%%%%%%%%%%%%%%%%%%%%%%%%%%%%%%%%%%%%%%%%%%%%%%%%%%
%%  左側に線を引く                                  %%
%%%%%%%%%%%%%%%%%%%%%%%%%%%%%%%%%%%%%%%%%%%%%%%%%%%%%%%
%leftbar環境の定義
\makeatletter
\renewenvironment{leftbar}{%
%  \def\FrameCommand{\vrule width 3pt \hspace{10pt}}%  デフォルトの線の太さは3pt
  \renewcommand\FrameCommand{\vrule width 1pt \hspace{10pt}}%
  \MakeFramed {\advance\hsize-\width \FrameRestore}}%
 {\endMakeFramed}
\newcommand{\exbf}[2]{ \begin{leftbar} \textbf{#1} #2 \end{leftbar} }%左線つき太字
%\newcommand{\barquo}[1]{\begin{leftbar} \begin{quo} #1 \end{quo} \end{leftbar}}%左線つき引用%びふぉあ
\newcommand{\barquo}[1]{\begin{leftbar} \noindent #1  \end{leftbar}}%左線つき引用
\newcommand{\lbar}[1]{\begin{leftbar} #1 \end{leftbar}}



%%%%%%%%%%%%%%%%%%%%%%%%%%%%%%%%%%%%%%%%%%%%%%%%%%%%%%%
%%  色をつける                                      %%
%%%%%%%%%%%%%%%%%%%%%%%%%%%%%%%%%%%%%%%%%%%%%%%%%%%%%%%
\newcommand{\textblue}[1]{\textcolor{blue}{\textbf{#1}}}

%%%%%%%%%%%%%%%%%%%%%%%%%%%%%%%%%%%%%%%%%%%%%%%%%%%%%%%
%%  よく使う記号の略記                                 %%
%%%%%%%%%%%%%%%%%%%%%%%%%%%%%%%%%%%%%%%%%%%%%%%%%%%%%%%
\newcommand{\setmid}[2]{\left\{ #1 \mathrel{} \middle| \mathrel{} #2 \right\}}%集合の内包記法
\newcommand{\sm}{\setminus}%集合差
\newcommand{\abs}[1]{\left \lvert #1 \right \rvert}%絶対値
\newcommand{\norm}[1]{\left \lVert #1 \right \rVert}%ノルム
\newcommand{\transpose}[1]{\, {\vphantom{#1}}^t\!{#1}}%行列の転置
\newcommand{\pmat}[1]{ \begin{pmatrix} #1 \end{pmatrix} }%まるかっこ行列
\newcommand{\f}[2]{\frac{#1}{#2}}%分数
\newcommand{\kakko}[1]{ \langle #1  \rangle}%鋭角かっこ%\angleはもうある
\newcommand{\I}{\sqrt{-1}}%虚数単位。\iは既にある。
\newcommand{\single}{\{ 0 \}}%0のシングルトン
\newcommand{\clsub}{\subset_{\text{closed}}}%閉部分集合
\newcommand{\opsub}{\subset_{\text{open}}}%開部分集合
\newcommand{\clirr}{\subset_{\text{closed irr}}}%閉既約部分集合
\newcommand{\loc}{\subset_{\text{loc. closed}}}%局所閉部分集合
\newcommand{\wt}[1]{\widetilde{#1}}%わいどちるだあ
\newcommand{\ol}[1]{\overline{#1}}%オーバーライン
\newcommand{\wh}[1]{\widehat{#1}}%ワイドハット
\newcommand{\To}{\Rightarrow}%ならば%自然変換
\newcommand{\xto}[1]{\xrightarrow}%上側文字付き右向き矢印
\newcommand{\st}{\; \; \text{s.t.} \; \;}%空白付きsuch that
\newcommand{\ts}{\otimes}%テンソル積
\newcommand{\tm}{\times}%直積
\newcommand{\vartm}{\times^{\text{Var}}}%多様体の圏における直積。集合の直積と区別するとき用。
\newcommand{\la}{\overleftarrow}%上付き左矢印
\newcommand{\ra}{\overrightarrow}%上付き右矢印
\newcommand{\del}{\partial}%偏微分の記号
\newcommand{\PD}[2]{  \frac{\partial #1}{\partial #2}  }%偏微分でるでる
\newcommand{\zyu}[1]{ \mathbb{Z} / #1 \mathbb{Z} }%有限巡回群
\newcommand{\s}[1]{ \sqrt{#1} }%平方根記号
\newcommand{\sh}{\mathbin{\sharp}}%シャープ%良い子の諸君!このシャープは二項演算として定義したので、ただのシャープではないことに注意だ
\newcommand{\id}{\mathrm{id}}%アイデンティティ



%%%%%%%%%%%%%%%%%%%%%%%%%%%%%%%%%%%%%%%%%%%%%%%%%%%%%%%
%%       演算子                                       %%
%%%%%%%%%%%%%%%%%%%%%%%%%%%%%%%%%%%%%%%%%%%%%%%%%%%%%%%
%log型
\DeclareMathOperator{\rank}{rank}%行列の階数
\DeclareMathOperator{\rk}{rk}%行列の階数
\DeclareMathOperator{\corank}{corank}%行列の核の次元
\renewcommand{\Re}{\operatorname{Re}}%実部
\DeclareMathOperator{\Res}{Res}%留数
\DeclareMathOperator{\Gal}{Gal}%Galois群
\DeclareMathOperator{\Hom}{Hom}%射の集合
\DeclareMathOperator{\ind}{Ind}
\DeclareMathOperator{\tr}{tr}%トレース
\DeclareMathOperator{\Tr}{Tr}%トレース
\DeclareMathOperator{\Norm}{N}%ノルム
\DeclareMathOperator{\Aut}{Aut}%自己同型群
\DeclareMathOperator{\trdeg}{tr\text{.}deg}%超越次数
\DeclareMathOperator{\Frac}{Frac}%商体をとる操作
\renewcommand{\Im}{\operatorname{Im}}%写像の像。Abel圏の像対象。虚部が出力できなくなった。
\DeclareMathOperator{\Ker}{Ker}%写像の核。Abel圏の核対象。
\DeclareMathOperator{\im}{im}%写像の像
\DeclareMathOperator{\coker}{coker}%余核%対象のほう
\DeclareMathOperator{\Coker}{Coker}%余核%射のほう
\DeclareMathOperator{\Spec}{Spec}%スペクトル
\DeclareMathOperator{\Sing}{Sing}%Singular point.特異点の集合。歌ってるわけではないぞ
\DeclareMathOperator{\Supp}{Supp}%台
\DeclareMathOperator{\ann}{ann}%アナイアレーター
\DeclareMathOperator{\Ass}{Ass}%素因子
\DeclareMathOperator{\ord}{ord}%おーだー
\DeclareMathOperator{\height}{ht}%素イデアルの高度。\htはもうある
\DeclareMathOperator{\coht}{coht}%素イデアルの余高度
\DeclareMathOperator{\Lan}{Lan}%左Kan拡張
\DeclareMathOperator{\Ran}{Ran}%右Kan拡張
\DeclareMathOperator{\Orbit}{Orbit}%作用の軌道
\DeclareMathOperator{\Stab}{Stab}%作用の安定化群
\DeclareMathOperator{\sgn}{sgn}%符号


%limit型
\DeclareMathOperator*{\llim}{\varprojlim}%極限。逆極限。射影極限。
\DeclareMathOperator*{\rlim}{\varinjlim}%余極限。順極限。入射極限。

%%%%%%%%%%%%%%%%%%%%%%%%%%%%%%%%%%%%%%%%%%%%%%%%%%%%%%%
%%  黒板太字(blackboard bold)                         %%
%%%%%%%%%%%%%%%%%%%%%%%%%%%%%%%%%%%%%%%%%%%%%%%%%%%%%%%
\newcommand{\bba}{{\mathbb A}}
\newcommand{\bbb}{{\mathbb B}}
\newcommand{\bbc}{{\mathbb C}}
\newcommand{\bbd}{{\mathbb D}}
\newcommand{\bbe}{{\mathbb E}}
\newcommand{\bbf}{{\mathbb F}}
\newcommand{\bbg}{{\mathbb G}}
\newcommand{\bbh}{{\mathbb H}}
\newcommand{\bbi}{{\mathbb I}}
\newcommand{\bbj}{{\mathbb J}}
\newcommand{\bbk}{{\mathbb K}}
\newcommand{\bbl}{{\mathbb L}}
\newcommand{\bbm}{{\mathbb M}}
\newcommand{\bbn}{{\mathbb N}}
\newcommand{\bbo}{{\mathbb O}}
\newcommand{\bbp}{{\mathbb P}}
\newcommand{\bbq}{{\mathbb Q}}
\newcommand{\bbr}{{\mathbb R}}
\newcommand{\bbs}{{\mathbb S}}
\newcommand{\bbt}{{\mathbb T}}
\newcommand{\bbu}{{\mathbb U}}
\newcommand{\bbv}{{\mathbb V}}
\newcommand{\bbw}{{\mathbb W}}
\newcommand{\bbx}{{\mathbb X}}
\newcommand{\bby}{{\mathbb Y}}
\newcommand{\bbz}{{\mathbb Z}}

%%%%%%%%%%%%%%%%%%%%%%%%%%%%%%%%%%%%%%%%%%%%%%%%%%%%%%%
%%  よく使う黒板太字                                  %%
%%%%%%%%%%%%%%%%%%%%%%%%%%%%%%%%%%%%%%%%%%%%%%%%%%%%%%%
\newcommand{\Z}{\bbz}%整数
\newcommand{\A}{\bba}
\newcommand{\Q}{\bbq}%有理数
\newcommand{\R}{\bbr}%実数
\newcommand{\C}{\bbc}%複素数
\newcommand{\F}{\bbf}%有限体
\newcommand{\N}{\bbn}%自然数
\newcommand{\T}{\bbt}%トーラス
\renewcommand{\P}{\bbp}%パラグラフ記号が出力できなくなった%射影空間

%%%%%%%%%%%%%%%%%%%%%%%%%%%%%%%%%%%%%%%%%%%%%%%%%%%%%%%
%%  カリグラフィー                                %%
%%%%%%%%%%%%%%%%%%%%%%%%%%%%%%%%%%%%%%%%%%%%%%%%%%%%%%%
%大文字しかどうせ使わない
\newcommand{\cala}{\mathcal{A}}
\newcommand{\calb}{\mathcal{B}}
\newcommand{\calc}{\mathcal{C}}
\newcommand{\cald}{\mathcal{D}}
\newcommand{\calf}{\mathcal{F}}
\newcommand{\calo}{\mathcal{O}}

%%%%%%%%%%%%%%%%%%%%%%%%%%%%%%%%%%%%%%%%%%%%%%%%%%%%%%%
%%  ギリシャ文字(Greek letters)小文字                 %%
%%%%%%%%%%%%%%%%%%%%%%%%%%%%%%%%%%%%%%%%%%%%%%%%%%%%%%%
%コマンドが5字以上のもの
\newcommand{\gra}{{\alpha}}
\newcommand{\grg}{{\gamma}}
\newcommand{\grd}{{\delta}}
\newcommand{\gre}{{\epsilon}}
\newcommand{\grt}{{\theta}}
\newcommand{\grk}{{\kappa}}
\newcommand{\grl}{{\lambda}}
\newcommand{\grs}{{\sigma}}
\newcommand{\gru}{{\upsilon}}
\newcommand{\gro}{{\omega}}

\newcommand{\ve}{{\varepsilon}}
\newcommand{\vp}{{\varphi}}

%%%%%%%%%%%%%%%%%%%%%%%%%%%%%%%%%%%%%%%%%%%%%%%%%%%%%%%
%%  ギリシャ文字(Greek letters)大文字                 %%
%%%%%%%%%%%%%%%%%%%%%%%%%%%%%%%%%%%%%%%%%%%%%%%%%%%%%%%
%コマンドが5字以上のもの
\newcommand{\grG}{{\Gamma}}
\newcommand{\grD}{{\Delta}}
\newcommand{\grT}{{\Theta}}
\newcommand{\grL}{{\Lambda}}
\newcommand{\grS}{{\Sigma}}
\newcommand{\grU}{{\Upsilon}}
\newcommand{\grO}{{\Omega}}

%%%%%%%%%%%%%%%%%%%%%%%%%%%%%%%%%%%%%%%%%%%%%%%%%%%%%%%
%%  フラクトゥール                                  %%
%%%%%%%%%%%%%%%%%%%%%%%%%%%%%%%%%%%%%%%%%%%%%%%%%%%%%%%
\newcommand{\fraka}{\mathfrak{a}}
\newcommand{\frakb}{\mathfrak{b}}
\newcommand{\frakm}{\mathfrak{m}}
\newcommand{\frakn}{\mathfrak{n}}
\newcommand{\frakp}{\mathfrak{p}}
\newcommand{\frakq}{\mathfrak{q}}

\newcommand{\frakA}{\mathfrak{A}}
\newcommand{\frakB}{\mathfrak{B}}
\newcommand{\frakS}{\mathfrak{S}}
\newcommand{\frakT}{\mathfrak{T}}

\newcommand{\Top}{\mathfrak{Top}}%開部分集合全体のなす有向集合
\newcommand{\Ab}{\mathfrak{Ab}}%Abel群のなす圏

%%%%%%%%%%%%%%%%%%%%%%%%%%%%%%%%%%%%%%%%%%%%%%%%%%%%%%%
%%  花文字                                          %%
%%%%%%%%%%%%%%%%%%%%%%%%%%%%%%%%%%%%%%%%%%%%%%%%%%%%%%%
%大文字しかどうせ使わない
\newcommand{\scra}{\mathscr{A}}
\newcommand{\scrf}{\mathscr{F}}
\newcommand{\scrg}{\mathscr{G}}
\newcommand{\scrh}{\mathscr{H}}
\newcommand{\scrs}{\mathscr{S}}

%%%%%%%%%%%%%%%%%%%%%%%%%%%%%%%%%%%%%%%%%%%%%%%%%%%%%%%
%%  太字                                            %%
%%%%%%%%%%%%%%%%%%%%%%%%%%%%%%%%%%%%%%%%%%%%%%%%%%%%%%%
\newcommand{\Sh}{\textbf{Sh}}%層の圏
\newcommand{\PSh}{\textbf{PSh}}%前層の圏
\newcommand{\bfzero}{\textbf{0}}%太字のゼロ

%小文字
\newcommand{\bfb}{\textbf{b}}
\newcommand{\bfu}{\textbf{u}}
\newcommand{\bfv}{\textbf{v}}
\newcommand{\bfx}{\textbf{x}}
\newcommand{\bfy}{\textbf{y}}


%大文字
\newcommand{\bfC}{\textbf{C}}
\newcommand{\bfD}{\textbf{D}}
\newcommand{\bfE}{\textbf{E}}
